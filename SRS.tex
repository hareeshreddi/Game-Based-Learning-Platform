\documentclass[a4paper,11pt]{article}
\usepackage[english]{babel}
\newcommand{\sectionbreak}{\clearpage}
\usepackage{titlesec}
\setcounter{tocdepth}{4}
\setcounter{secnumdepth}{4}
\setcounter{tocdepth}{5}
\setcounter{secnumdepth}{5}

\begin{document}
\title{Software Requirements Specifications Document\\\(Game Based Learning\)\\Project-4}
\author{150101027- I.N.Dilip Kumar\\ 150101051- Hareesh Reddi\\ 150101032- L.Sai Shobith}
\date{\today}
\maketitle
\tableofcontents{}
\section{Introduction}

\subsection{Purpose}
	The purpose of this document is to present a detailed description about the “Game Based Learning”. This document will explain the purpose and features of the system, Interfaces of the
system, Various constraints under which the system has to operate and requirement analysis which includes both the functional and nonfunctional requirements along with external interface requirements and performance requirements.

\subsection{Definitions}
\begin{tabular}{|c|l|}
  \hline			
  Term & Definition \\ \hline
  Game Engine &A game engine is a system designed for the creation and development of \\&video games \\ &\\ \hline
 Android & Android (Google product) is a Linux-based operating system \\ &\\ \hline
SRS & Software Requirements Specification \\ &\\ \hline
UI & User Interface \\ &\\ \hline
Gamer & A person who plays a game or games, typically a participant in a computer \\&or role-playing game \\ &\\ \hline
System & A system is a set of interacting or interdependent components forming an\\& integrated whole or a set of elements (often called ‘components’) and 
\\&relationships which are different from relationships of the set or its\\& elements to other elements or sets. \\  \hline  
\end{tabular}

\subsection{Scope of the Project}
	The Mobile App will be a platform especially for students who are pursuing their under graduation. This app will help them understand the sorting algorithms like Insertion sort, Quick sort and Bubble sort in a more efficient way which is in a game based learning platform. \\ \\The main intention of ours is to provide the users/students with a mobile app through which the user will be able to increase their extent of un-derstanding about the algorithm and therefore can improve their clarity about these concepts. More specifically the student/user will be able to improve his/her performance by looking at their Statistical data which is provided by the app.

\subsection{Intended Audience and Reading Suggestions}
	The SRS document gives project managers a way to ensure the game’s adherence to our original vision. Although the document may be read from front to back for a complete understanding of the project, it was written in sections and hence can be read as such. For an overview of the document and the project itself, see Overall Description.

\subsection{Overview of the Document}
	The first chapter contains the Introduction part which gives a basic idea about what the project is going to be and provides the definitions which will be used throughout the document and also the basic definitions given by the IEEE Computer society. 

	The second chapter will give the overall description which contains all the interfaces like system interface, user interface, software and hardware interfaces in a detailed manner followed by Design constraints, Product function, User characteristics and will be ended with the Memory constraints.

	The third chapter will have the system requirements which include functional requirements and non-functional requirements like security, maintainability, portability.

\subsection{References}
\begin{enumerate}
\item https://en.wikipedia.org/wiki/Software\_requirements\_specification\#Structure
\item Software Engineering: A Practitioner's Approach, Roger S Pressman - sixth Edition
\item http://www.nptel.ac.in/courses/106105087/
\item https://www.cise.ufl.edu/class/cen3031sp13/SRS\_Example\_1\_2011.pdf
\item IEEE SRS Template
\end{enumerate}
\section{Overall Description}
\subsection{Productive perspective}
The app is supposed to be an android based game highly useful for students especially undergraduates. The product is independent and self-contained. \\
	The following are the main features:
\begin{itemize}
\item This app provides a very efficient way to understand the sorting algorithms.
\item This app gives a statistical report at the end of completion of each game, so the user can know whether he is improving or not.
\item The student can play and learn during the lecture or outside of the college hours.
\item The teacher can also use this as a supplementary for teaching sorting algorithms.
\end{itemize}
\subsubsection{User Interface}
\begin{itemize}
\item Every game must have a menu so it is easy for the user to see everything at one spot.
\item The user on opening the application can use the various utilities of the app like Start Game, Highscores.
\end{itemize}
\subsubsection{Hardware Interfaces}
\begin{itemize}
\item Touch screen
\item Speakers to generate sounds
\end{itemize}
\subsubsection{Software Interfaces}
\begin{itemize}
\item Product should run on android versions 4.4 Kitkat and after
\item Product is preferably made on Unity-3D, Blender using C\#.
\item C\# programming language: C\# is a hybrid of C and C++. It is an object oriented programming language used with XML- based web services on .NET platform. It's also used by several type of system for building other applications. Examples that use C\# are visual studio, Unity etc.
\item Unity-3D: Unity is a cross platform game engine developed by Unity Technologies and used to develop video games. We will use Unity-3D to build the game and then publish it to Android.
\item Blender: Blender is an open-source 3D graphic software product used to create animated video, 3D printed model, interactive 3D application and video games. 
\item SQLite for database
\end{itemize}
\subsubsection{Communication Interface}
The communication between the different parts of the system is very important since they depend on each other. However, in what way the communication is achieved is not important for the system and is therefore handled by the underlying operating systems. There will not be any communications between this app and the other apps since it is completely independent and self contained.

\subsection{User Characteristics}
There is only one user at a time in this software and the user interacts with the game/system. 
	Therefore, the user is the only one who communicates with the system through playing the game. In addition, this gamer can be any person. The only primary requirement is that, the player who is playing must read the playing procedure and instructions provided by the developers.

\subsection{Assumptions and Dependencies}
One assumption about the product is that it will always be used on mobile phones that have enough performance. If the phone does not have enough hardware resources available for the application, for example, the users might have allocated them with other applications; there may be scenarios where the application does not work as intended or even at all. 
	Another assumption is that the user who uses this app is acquainted with enough knowledge of how to use an android mobile phone.

\section{Specific Requirements}
\subsection{Functional Requirements}
\subsubsection{Player Name} 
\textbf{Input}: String (Name of the player)\\
\textbf{Output}: Player name is stored for later usage.\\
\textbf{Description}: The user who is playing the game enters his/her name. The final score of the game played by the user is saved in the high scores database with this name.

\subsubsection{Start Game} 
\textbf{Input}: Input wil be taken from the touch screen\\
\textbf{Output}: Game will be started\\
\textbf{Description}: The system shall allow the player to start a game.

\paragraph{Random Number Generator}\hfill \break
\textbf{Input}: Number of two digit numbers to be generated\\
\textbf{Output}: Two digit numbers in unsorted order.\\
\textbf{Description}: The function gives the requested number of two digit numbers in unsorted order, so that user can sort them.

\subparagraph{User Moves Correctness}\hfill \break
\textbf{Input}: The move performed by the player\\
\textbf{Output}: Score is increased if the move is correct and score is decreased if the move in incorrect.\\
\textbf{Description}: This function determines whether the player performed a correct or incorrect move and increases/decreases the score in score board accordingly.

\subparagraph{Three Consecutive Wrong Moves}\hfill \break
\textbf{Input}: Last three moves\\
\textbf{Output}: Restarts the game if not all the given moves are correct.\\
\textbf{Description}: If the player performs three consecutive wrong moves then the game is restarted.

\paragraph{Pause Game}\hfill \break
\textbf{Input}: Input will be taken from the touch screen\\
\textbf{Output}: Time counter stops. User cannot make any moves in the game.\\
\textbf{Description}: A player can pause a game, so that he can resume it afterwards.

\subparagraph{Resume Game}\hfill \break
\textbf{Input}: Input will be taken from the touch screen\\ 
\textbf{Output}: Time counter starts. User can make moves in the game\\
\textbf{Description}: A player can resume a paused game.

\paragraph{Restart Game}\hfill \break
\textbf{Input}: Input will be taken from the touch screen\\
\textbf{Output}: Starts a new game\\
\textbf{Description}: The player can restart in the middle of an ongoing game or when there are three consecutive wrong moves, the user will be asked to restart the game.
\paragraph{Stop Game}\hfill \break
\textbf{Input}: Input will be taken from the touch screen\\
\textbf{Output}: Stops the current game.\\
\textbf{Description}: The current game the user is playing is aborted.

\paragraph{Time Counter}\hfill \break
\textbf{Input}: None\\
\textbf{Output}: Duration of the game played\\
\textbf{Description}: The total time taken by the player to complete a game is calculated

\subsubsection{Game Over}
\textbf{Input}: The array of numbers\\ 
\textbf{Output}: Stops the game\\
\textbf{Description}: The game is over if the numbers are in sorted order.

\paragraph{Score Generator}\hfill \break
\textbf{Input}: No of wrong moves, number of right moves and time taken.\\
\textbf{Output}: Total score of the player in the game\\
\textbf{Description}: This function calculates the final score by considering number of wrong moves, number of right moves and the time taken by the player to finish the game and gives the average time to complete the game. 

\paragraph{High Scores}\hfill \break
\textbf{Input}: Total score of the player in the game.\\
\textbf{Output}: Score is stored in the game if it is high enough\\
\textbf{Description}: If the score of the player is higher than at least one of the high score or the high score, database is not full then it is stored in the high scores

\subsubsection{Exit}
\textbf{Input}: Input will be from the touch screen\\
\textbf{Output}: The game application ends.\\
\textbf{Description}: The game app stops running.

\subsection{Performance Requirements}
\begin{itemize}
\item The minimum frame rate must be twenty frames per second.  The average frame rate must be greater than thir-ty.
\item The game must be able to run with a minimum of 512 MB of RAM.
\item The average response time between click and reaction must be less than 0.5 seconds. The maximum response time between click and reaction must not be greater 2 seconds.
\end{itemize}
\subsection{Non-Functional Requirements}
\subsubsection{Reliability}
		As the system provide the right tools for discussion and problem solving, it must be made sure that the system is reliable in its operations.
\subsubsection{Usability}
		The app is easy to handle and navigates in a most expected way with no delays. A user help guide is also constructed to give the user a complete documented version of what the app is and how can he/she make it useful to the extreme levels.
\subsubsection{Security}
		The data which is generated in the app is not necessarily need to be secured. because the data generated will be like the Name of the player, number of restart he/she has used in the game, Average time needed to complete the game. So, it is not needed to secure this data and can be viewed by anyone who opens the app. But the data generated in this app cannot be accessed from another apps.
\subsubsection{Maintainability}
\begin{itemize}
\item The app should ask to send the bug reports and crash reports to the developer which would help in maintenance.
\item The app should ask for feedbacks so as to improve the app in next updates.
\item The app shall be updated frequently.
\item The source will be properly commented such that it may be maintained by others in future.
\end{itemize}
\subsubsection{Portability}
		The app should be easily ported into any other device of android platform.An .apk file will be made available to install the application on that device.
\subsubsection{Extensibility}
		The app should be made such that it may be extended to a larger system which has more number of sorting algorithms.
\end{document}